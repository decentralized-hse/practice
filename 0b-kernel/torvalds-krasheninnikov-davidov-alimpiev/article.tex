\documentclass[12pt,a4paper]{article}

% ===== Языки и шрифты =====
\usepackage[T2A]{fontenc}
\usepackage[utf8]{inputenc}
\usepackage[english,russian]{babel}

% ===== Поля =====
\usepackage[left=2.5cm, right=2.5cm, top=2.5cm, bottom=2.5cm]{geometry}

% ===== Графика и таблицы =====
\usepackage{graphicx}
\usepackage{booktabs}
\usepackage{float}
\usepackage{subcaption}
\usepackage{tabularx}

% ===== Ссылки =====
\usepackage{hyperref}
\hypersetup{
    colorlinks=true,
    linkcolor=blue,
    citecolor=blue,
    urlcolor=blue
}

% ===== Прочее =====
\usepackage{amsmath}
\usepackage{csquotes}

% ===== Метаданные =====
\title{Следы создателя: географический и структурный анализ кода\\
Линуса Торвальдса в ядре Linux}
\author{
    technothecow,
    Quizert,
    LevAlimpiev \\[6pt]
    \small Национальный исследовательский университет \\
    \small <<Высшая школа экономики>>
}
\date{Февраль 2026}

\begin{document}
\maketitle

\begin{abstract}
    Ядро Linux~--- один из крупнейших проектов с открытым исходным кодом, объединяющий
    тысячи разработчиков. Однако код его создателя, Линуса Торвальдса, по-прежнему
    присутствует в~кодовой базе и~эволюционирует вместе с~ней.
    В~данной работе мы анализируем не \emph{объём}, а \emph{характер и~локализацию}
    кода Торвальдса: в~каких подсистемах он сосредоточен, на каких языках написан,
    как меняется его <<география>> от версии к~версии. Анализ проводится по данным
    \texttt{git~blame} для четырёх знаковых версий ядра (3.0, 4.15, 5.10, 6.18),
    охватывающих период с~2011 по~2025~год.
    Результаты показывают, что абсолютный след Торвальдса неуклонно сокращается
    (с~3,5~млн до~2,0~млн строк), однако его код остаётся стабильно распределён
    между подсистемами, а~доля периферии (\texttt{drivers/}, \texttt{sound/})
    составляет $\sim$55\% во~всех версиях.
\end{abstract}

% ==========================================
\section{Введение}
% ==========================================

Ядро Linux, первая версия которого была опубликована Линусом Торвальдсом в~1991~году,
к~настоящему моменту является одним из самых масштабных совместных программных проектов
в~мире. По данным Linux Foundation, в~разработке ядра приняли участие более 20~000
разработчиков из более чем 1~700 компаний~\cite{linuxfoundation2020}.

Традиционный подход к~анализу авторства кода сосредоточен на количественных
показателях: \emph{сколько} строк написал тот или иной разработчик. Однако
в~проекте масштаба ядра Linux такой подход упускает важную качественную
составляющую: \emph{где} именно в~архитектуре системы сосредоточен код автора,
\emph{какого} он типа (исходный код, заголовки, ассемблер, конфигурация), и~как
эта картина меняется во~времени.

В~данной работе мы смещаем фокус с~количества на структуру и~локализацию.
Мы~анализируем только те файлы ядра, в~которых по данным \texttt{git blame}
присутствуют строки авторства Торвальдса, и~исследуем:
\begin{itemize}
    \item в каких подсистемах ядра сосредоточен его код;
    \item на каких языках и в каких типах файлов он пишет;
    \item насколько концентрирован или распределён его вклад;
    \item как <<след создателя>> мигрирует между подсистемами от версии к~версии.
\end{itemize}

% ==========================================
\section{Обоснование выбора версий}
\label{sec:versions}
% ==========================================

Для анализа выбраны четыре версии ядра, каждая из которых маркирует значимый этап
в~эволюции проекта (табл.~\ref{tab:versions}).

\begin{table}[H]
    \centering
    \caption{Выбранные версии ядра и обоснование}
    \label{tab:versions}
    \begin{tabularx}{\textwidth}{llX}
        \toprule
        \textbf{Версия} & \textbf{Дата} & \textbf{Обоснование} \\
        \midrule
        3.0 & Июль 2011 &
            Символический <<перезапуск>> нумерации после серии 2.6.x (2003--2011).
            Удобная отправная точка: ядро уже использует Git (с~2005~г.),
            сообщество выросло, а~эффект начальной миграции на Git~--- когда
            Торвальдс формально являлся автором $\sim$80\% коммитов~---
            уже~сгладился. \\[4pt]
        4.15 & Январь 2018 &
            Версия, вышедшая сразу после обнаружения уязвимостей Meltdown и~Spectre.
            Масштабные патчи безопасности от Intel, Google, ARM и~др. привели к
            всплеску активности сторонних контрибьюторов, особенно в~подсистемах
            \texttt{arch/} и~\texttt{kernel/}. \\[4pt]
        5.10 & Декабрь 2020 &
            Версия с~долгосрочной поддержкой (LTS), выпущенная в~разгар
            пандемии COVID-19. Одна из ключевых LTS-версий, широко используемая
            в~промышленных системах (Android, облачные платформы). \\[4pt]
        6.18 & Май 2025 &
            Одна из последних стабильных версий на момент исследования.
            Позволяет оценить текущее состояние проекта;
            роль Торвальдса сместилась к~управлению и~архитектурным решениям. \\
        \bottomrule
    \end{tabularx}
\end{table}

Выбор именно версий, а не произвольных дат, обусловлен тем, что каждый тег версии
в~Git соответствует конкретному стабильному состоянию кодовой базы, пригодному для
воспроизводимого анализа. Период между версиями неравномерен,
но это сознательный выбор: нас интересуют не временны\'{е} интервалы, а~качественные
изменения в~проекте.

% ==========================================
\section{Методология}
\label{sec:methodology}
% ==========================================

\subsection{Сбор данных}

Для каждой из четырёх версий из репозитория ядра Linux извлекались данные
\texttt{git blame} с~фильтрацией по автору \texttt{Linus Torvalds}. Для каждого файла,
содержащего хотя бы одну строку авторства Торвальдса, фиксировались:
\begin{itemize}
    \item путь к файлу (\texttt{file\_path});
    \item общее количество строк в~файле (\texttt{total\_lines});
    \item количество строк авторства Торвальдса (\texttt{linus\_lines}).
\end{itemize}

\noindent Результаты агрегированы в~CSV-файл формата:
\begin{verbatim}
snapshot_date,file_path,linus_lines,total_lines
21-07-2011,kernel/sched/core.c,150,8400
21-07-2011,mm/memory.c,45,3200
...
\end{verbatim}

\noindent Суммарный объём данных: 42~002 записи по~15~626 уникальным файлам в~четырёх версиях.

\textbf{Важное ограничение.} В~выборку включены \emph{только файлы, содержащие код
Торвальдса}. Это означает, что мы не можем корректно вычислить долю Торвальдса
среди \emph{всех} файлов ядра. Однако для целей нашего исследования это ограничение
не~является недостатком: нас интересует не абсолютная доля, а~\emph{структура и
характер} его вклада~--- в~каких подсистемах, файлах и на каких языках
сосредоточен его код.

\textbf{Второе ограничение: эффект миграции на~Git.} Ядро Linux перешло на систему
контроля версий Git в~2005~году. При~миграции большая часть кода была
зафиксирована в~первом коммите с~Линусом Торвальдсом в~качестве автора,
даже если этот код был написан другими разработчиками в~более ранних системах
контроля версий (BitKeeper, CVS). Это означает, что часть кода, атрибутированного
Торвальдсу по данным \texttt{git blame}, на~самом деле не~является его авторством.
Для~точного анализа \emph{истинного} вклада Торвальдса необходим анализ
диффов между версиями, что выходит за~рамки данной работы.

\subsection{Извлечение признаков}

Из~пути к~файлу (\texttt{file\_path}) извлекаются следующие признаки:
\begin{itemize}
    \item \textbf{Подсистема}~--- директория верхнего уровня
    (\texttt{kernel/}, \texttt{mm/}, \texttt{fs/}, \texttt{drivers/},
    \texttt{net/}, \texttt{arch/}, \texttt{include/} и~др.).
    \item \textbf{Тип файла}~--- расширение:
    \texttt{.c}~(исходный код),
    \texttt{.h}~(заголовки),
    \texttt{.S}~(ассемблер),
    \texttt{Kconfig}~(конфигурация),
    \texttt{Makefile} и~др.
    \item \textbf{Глубина пути}~--- количество компонентов в~пути
    (например, \texttt{kernel/sched/core.c} имеет глубину~3).
\end{itemize}

\subsection{Метрики}

На~основе собранных данных вычисляются следующие группы метрик.

\subsubsection{Метрики масштаба присутствия}
\begin{itemize}
    \item $N_{\text{files}}(v)$~--- количество файлов с~кодом Торвальдса в~версии~$v$;
    \item $L_{\text{total}}(v) = \sum_f \text{linus\_lines}(f, v)$~--- суммарное число
    строк Торвальдса;
    \item $N_{\text{subsystems}}(v)$~--- количество подсистем (директорий верхнего
    уровня), в~которых присутствует его код.
\end{itemize}

\subsubsection{Метрики концентрации}
\begin{itemize}
    \item \textbf{Доля в~файле}: $p(f, v) = \text{linus\_lines}(f,v) / \text{total\_lines}(f,v)$~---
    для каждого файла;
    \item \textbf{Индекс концентрации Герфиндаля--Хиршмана (HHI)} по подсистемам:
    \[
        \text{HHI}(v) = \sum_{s} \left(\frac{L_s(v)}{L_{\text{total}}(v)}\right)^{\!2},
    \]
    где $L_s(v)$~--- число строк Торвальдса в~подсистеме~$s$. HHI близкий к~1
    означает, что код сконцентрирован в~одной подсистеме; низкий HHI~--- что
    код распределён равномерно.
\end{itemize}

\subsubsection{Метрики распределения по типам}
\begin{itemize}
    \item Доля строк в~файлах \texttt{.c}, \texttt{.h}, \texttt{.S},
    и прочих~--- по каждой версии.
\end{itemize}

\subsubsection{Структурные метрики}
\begin{itemize}
    \item \textbf{<<Ядро>> vs <<периферия>>}: доля строк в~ключевых
    подсистемах (\texttt{kernel/}, \texttt{mm/}, \texttt{fs/}, \texttt{init/},
    \texttt{ipc/}, \texttt{lib/}, \texttt{block/})
    против периферийных (\texttt{drivers/}, \texttt{sound/}, \texttt{samples/},
    \texttt{tools/}).
\end{itemize}

% ==========================================
\section{Результаты}
\label{sec:results}
% ==========================================

\subsection{Масштаб присутствия}

\begin{table}[H]
    \centering
    \caption{Масштаб присутствия кода Торвальдса по версиям}
    \label{tab:scale}
    \begin{tabular}{lcccc}
        \toprule
        & \textbf{3.0} & \textbf{4.15} & \textbf{5.10} & \textbf{6.18} \\
        \midrule
        Файлов с~кодом Торвальдса     & 11\,778 & 10\,850 & 9\,896 & 9\,478 \\
        Суммарно строк Торвальдса      & 3\,519\,576 & 2\,733\,204 & 2\,319\,651 & 1\,956\,144 \\
        Подсистем с~его кодом          & 19 & 20 & 21 & 23 \\
        Среднее строк/файл             & 298,8 & 251,9 & 234,4 & 206,4 \\
        Медиана строк/файл             & 115 & 88 & 75 & 57 \\
        \bottomrule
    \end{tabular}
\end{table}

Абсолютный <<след>> Торвальдса монотонно сокращается: количество файлов упало на~19,5\%
(с~11\,778 до~9\,478), суммарное число его строк~--- на~44,4\% (с~3,52~млн до~1,96~млн).
При~этом число подсистем, в~которых присутствует его код, выросло с~19 до~23~---
его <<географический охват>> расширился, несмотря на~сокращение объёма.


\subsection{География кода: распределение по подсистемам}

В~таблице~\ref{tab:top10_subsystems} представлены десять подсистем с~наибольшим
числом строк Торвальдса для каждой версии.

\begin{table}[H]
    \centering
    \caption{Топ-10 подсистем по числу строк Торвальдса}
    \label{tab:top10_subsystems}
    \small
    \begin{tabular}{cl@{\hskip6pt}r@{\hskip10pt}cl@{\hskip6pt}r}
        \toprule
        \multicolumn{3}{c}{\textbf{Версия 3.0}} & \multicolumn{3}{c}{\textbf{Версия 4.15}} \\
        \cmidrule(r){1-3}\cmidrule(l){4-6}
        \# & Подсистема & Строки & \# & Подсистема & Строки \\
        \midrule
        1 & drivers  & 1\,716\,118 & 1 & drivers  & 1\,333\,912 \\
        2 & arch     & 737\,029    & 2 & arch     & 581\,634    \\
        3 & fs       & 346\,212    & 3 & fs       & 264\,357    \\
        4 & sound    & 241\,896    & 4 & sound    & 188\,729    \\
        5 & net      & 221\,787    & 5 & net      & 158\,571    \\
        6 & include  & 152\,201    & 6 & include  & 97\,176     \\
        7 & scripts  & 21\,363     & 7 & lib      & 30\,588     \\
        8 & kernel   & 17\,841     & 8 & scripts  & 19\,116     \\
        9 & mm       & 15\,601     & 9 & kernel   & 13\,081     \\
       10 & security & 12\,018     &10 & mm       & 10\,902     \\
        \bottomrule
    \end{tabular}

    \vspace{6pt}

    \begin{tabular}{cl@{\hskip6pt}r@{\hskip10pt}cl@{\hskip6pt}r}
        \toprule
        \multicolumn{3}{c}{\textbf{Версия 5.10}} & \multicolumn{3}{c}{\textbf{Версия 6.18}} \\
        \cmidrule(r){1-3}\cmidrule(l){4-6}
        \# & Подсистема & Строки & \# & Подсистема & Строки \\
        \midrule
        1 & drivers  & 1\,101\,655 & 1 & drivers  & 918\,467  \\
        2 & arch     & 462\,533    & 2 & arch     & 389\,253  \\
        3 & fs       & 245\,478    & 3 & fs       & 191\,310  \\
        4 & sound    & 180\,077    & 4 & sound    & 165\,340  \\
        5 & net      & 147\,856    & 5 & net      & 129\,973  \\
        6 & include  & 87\,581     & 6 & include  & 78\,928   \\
        7 & lib      & 29\,297     & 7 & lib      & 29\,191   \\
        8 & kernel   & 11\,419     & 8 & kernel   & 9\,327    \\
        9 & scripts  & 11\,104     & 9 & security & 8\,479    \\
       10 & mm       & 10\,191     &10 & scripts  & 8\,209    \\
        \bottomrule
    \end{tabular}
\end{table}

Во~всех четырёх версиях лидером остаётся \texttt{drivers/}, за ним~--- \texttt{arch/},
\texttt{fs/}, \texttt{sound/} и~\texttt{net/}. Порядок первых шести позиций
не~менялся ни разу; сдвиги наблюдаются только во~второй половине рейтинга.


\subsection{Типы файлов: на чём пишет Торвальдс}

\begin{table}[H]
    \centering
    \caption{Распределение строк Торвальдса по типам файлов}
    \label{tab:file_types}
    \begin{tabular}{lrrrr}
        \toprule
        \textbf{Тип} & \textbf{3.0} & \textbf{4.15} & \textbf{5.10} & \textbf{6.18} \\
        \midrule
        \texttt{.c} (исходный код)  & 2\,608\,391 & 2\,021\,948 & 1\,726\,922 & 1\,432\,135 \\
        \texttt{.h} (заголовки)     & 593\,158    & 435\,844    & 359\,084    & 303\,657    \\
        \texttt{.S} (ассемблер)     & 185\,951    & 161\,537    & 142\,246    & 132\,338    \\
        Прочие                      & 93\,633     & 85\,450     & 69\,463     & 70\,186     \\
        \texttt{Kconfig}            & 28\,678     & 20\,872     & 15\,920     & 12\,889     \\
        \texttt{Makefile}           & 9\,765      & 7\,553      & 6\,016      & 4\,939      \\
        \bottomrule
    \end{tabular}
\end{table}

Подавляющее большинство строк Торвальдса~--- это исходный код на~C (\texttt{.c}),
составляющий $\sim$73\% его вклада во~всех версиях. На~втором месте~--- заголовки
(\texttt{.h}), далее~--- ассемблер (\texttt{.S}). Конфигурационные файлы
(\texttt{Kconfig}, \texttt{Makefile}) занимают менее 2\% суммарно.


\subsection{Концентрация: индекс HHI по подсистемам}

\begin{table}[H]
    \centering
    \caption{Индекс концентрации Герфиндаля--Хиршмана по подсистемам}
    \label{tab:hhi}
    \begin{tabular}{lccc}
        \toprule
        \textbf{Версия} & \textbf{HHI} & \textbf{Строки Линуса} & \textbf{Подсистемы} \\
        \midrule
        3.0   & 0,302 & 3\,519\,576 & 19 \\
        4.15  & 0,302 & 2\,733\,204 & 20 \\
        5.10  & 0,288 & 2\,319\,651 & 21 \\
        6.18  & 0,283 & 1\,956\,144 & 23 \\
        \bottomrule
    \end{tabular}
\end{table}

HHI остаётся стабильным~--- около 0,3 во~всех версиях, с~незначительным снижением
от~3.0 к~6.18. Это означает, что распределение кода Торвальдса по подсистемам
\emph{не~меняется принципиально}, несмотря на~сокращение абсолютного объёма и~рост
числа подсистем.


\subsection{Ядро vs периферия}

\begin{table}[H]
    \centering
    \caption{Строки Торвальдса: ключевые подсистемы vs периферия}
    \label{tab:core_periph}
    \begin{tabular}{llrrl}
        \toprule
        \textbf{Версия} & \textbf{Группа} & \textbf{Строк} & \textbf{Файлов} & \textbf{Доля строк} \\
        \midrule
        3.0 & core       & 400\,045   & 1\,103 & 11,4\% \\
        3.0 & periphery  & 1\,958\,015 & 3\,998 & 55,6\% \\
        3.0 & other      & 1\,161\,516 & 6\,677 & 33,0\% \\
        \midrule
        4.15 & core       & 329\,192   & 1\,184 & 12,0\% \\
        4.15 & periphery  & 1\,522\,672 & 3\,863 & 55,7\% \\
        4.15 & other      & 881\,340   & 5\,803 & 32,2\% \\
        \midrule
        5.10 & core       & 304\,487   & 1\,173 & 13,1\% \\
        5.10 & periphery  & 1\,281\,770 & 3\,548 & 55,3\% \\
        5.10 & other      & 733\,394   & 5\,175 & 31,6\% \\
        \midrule
        6.18 & core       & 243\,753   & 1\,124 & 12,5\% \\
        6.18 & periphery  & 1\,083\,923 & 3\,691 & 55,4\% \\
        6.18 & other      & 628\,468   & 4\,663 & 32,1\% \\
        \bottomrule
    \end{tabular}
\end{table}

Доля периферии (\texttt{drivers/}, \texttt{sound/}, \texttt{tools/}) в~строках
Торвальдса устойчиво составляет $\sim$55\% во~всех версиях. Ядро
(\texttt{kernel/}, \texttt{mm/}, \texttt{fs/}, \texttt{init/}, \texttt{ipc/},
\texttt{lib/}, \texttt{block/}) занимает $\sim$11--13\%.

Этот результат \emph{противоречит} наивному ожиданию, что основатель
концентрируется на~<<ядре ядра>>. Объяснение: подсистема \texttt{drivers/}~---
самая крупная по объёму часть Linux, и~Торвальдс исторически являлся автором
значительной части раннего кода драйверов, который постепенно вытесняется,
но~всё ещё присутствует.


\subsection{Ключевые наблюдения}

\begin{enumerate}
    \item \textbf{Устойчивое сокращение.} Абсолютное число строк Торвальдса падает
    монотонно: $-$44\% за~период 3.0--6.18. Число файлов~--- аналогично ($-$20\%).
    При~этом среднее число строк на~файл тоже снижается (с~299 до~206), что говорит
    об~<<размывании>> его вклада внутри файлов.

    \item \textbf{Рост географии.} Число подсистем, в~которых присутствует код
    Торвальдса, выросло с~19 до~23, включая появление \texttt{rust/} в~версии~6.18.

    \item \textbf{Стабильная структура.} Ранжирование подсистем по объёму кода
    Торвальдса практически не~меняется: \texttt{drivers} $>$ \texttt{arch} $>$
    \texttt{fs} $>$ \texttt{sound} $>$ \texttt{net} $>$ \texttt{include} во~всех
    четырёх версиях. Индекс HHI стабилен ($\approx 0{,}3$).

    \item \textbf{Преобладание C.} Торвальдс~--- автор преимущественно
    C-кода (\texttt{.c} $\approx$ 73\%) и~заголовков (\texttt{.h} $\approx$ 16\%).
    Доля ассемблера составляет $\sim$5\%, остальные типы~--- менее 3\%.

    \item \textbf{Периферия $>$ ядро.} Вопреки ожиданиям, большая часть
    оставшегося кода Торвальдса находится в~периферии ($\sim$55\%), а~не~в~ключевых
    подсистемах ядра ($\sim$12\%).
\end{enumerate}


% ==========================================
\section{Обсуждение}
\label{sec:discussion}
% ==========================================

Полученные результаты позволяют составить <<портрет>> вклада основателя в~крупном
проекте с~открытым исходным кодом.

Главный вывод~--- \emph{форма} вклада Торвальдса удивительно стабильна при
драматическом сокращении его \emph{объёма}. Структура распределения по~подсистемам,
типам файлов и~соотношение <<ядро/периферия>> практически не~изменились
за~14 лет (3.0--6.18), несмотря на~то что абсолютное число его строк
сократилось почти вдвое.

Это наблюдение согласуется с~гипотезой о~\emph{пропорциональном вытеснении}:
код Торвальдса замещается другими контрибьюторами примерно равномерно
во~всех подсистемах, а~не~вытесняется из~одних подсистем в~пользу~других.

Неожиданным является высокая доля периферии ($\sim$55\%) в~его коде.
Объяснение кроется в~том, что \texttt{drivers/}~--- исторически самая
объёмная часть ядра, и~Торвальдс являлся автором значительной доли
раннего кода драйверов. Хотя этот код постепенно переписывается,
он~всё ещё составляет наибольшую абсолютную долю.

С~точки зрения программной инженерии, подобный паттерн~--- когда основатель
переходит от~широкого написания кода к~точечным вмешательствам,
а~его исторический вклад <<размывается>> равномерно~---
можно рассматривать как естественную стратегию масштабирования open-source проекта.

% ==========================================
\section{Заключение}
\label{sec:conclusion}
% ==========================================

В~данной работе мы провели структурный анализ кода Линуса Торвальдса в~ядре Linux
на~протяжении четырёх ключевых версий (3.0, 4.15, 5.10, 6.18). В~отличие от
традиционного количественного подхода, мы сосредоточились на~\emph{географии}
и~\emph{характере} кода: его распределении по~подсистемам, типам файлов
и~структуре <<ядро~vs~периферия>>.

Основные результаты:
\begin{itemize}
    \item Абсолютный след Торвальдса сокращается ($-$44\% строк за~14 лет),
    но~его структурное распределение остаётся стабильным (HHI $\approx 0{,}3$).
    \item Код преимущественно на~C и~в~заголовках; доминирующие подсистемы~---
    \texttt{drivers/}, \texttt{arch/}, \texttt{fs/}.
    \item Периферия устойчиво составляет $\sim$55\% его вклада.
\end{itemize}

Предложенная методология и~набор метрик могут быть применены для анализа
роли основателей в~других крупных open-source проектах.

\textbf{Направления для дальнейших исследований.}
Важно отметить, что данные \texttt{git blame} несут искажение, связанное
с~миграцией ядра на~Git в~2005~году: значительная часть кода, атрибутированного
Торвальдсу, могла быть написана другими разработчиками в~более ранних системах
контроля версий. Для~точного анализа \emph{истинного} вклада необходимо
исследование диффов между версиями (а~не~кумулятивного состояния), что позволит
отделить код, написанный Торвальдсом в~рамках Git-истории, от~кода, унаследованного
при~миграции. Такое исследование может стать ценным дополнением к~настоящей работе.

% ==========================================
% Список литературы
% ==========================================
\begin{thebibliography}{9}

\bibitem{linuxfoundation2020}
    Linux Foundation,
    \textit{2020 Linux Kernel History Report},
    2020.
    \url{https://www.linuxfoundation.org/resources/publications/linux-kernel-history-report-2020}

\bibitem{torvalds2005git}
    Torvalds, L.,
    \textit{The Git Version Control System},
    2005.
    \url{https://git-scm.com/}

\bibitem{kroah2006linux}
    Kroah-Hartman, G., Corbet, J., McPherson, A.,
    \textit{Linux Kernel Development: How Fast It is Going, Who is Doing It,
    What They are Doing, and Who is Sponsoring It},
    Linux Foundation, 2006--2020.

\bibitem{raymond1999cathedral}
    Raymond, E.~S.,
    \textit{The Cathedral and the Bazaar},
    O'Reilly Media, 1999.

\bibitem{mockus2002}
    Mockus, A., Fielding, R.~T., Herbsleb, J.~D.,
    \textit{Two Case Studies of Open Source Software Development:
    Apache and Mozilla},
    ACM Transactions on Software Engineering and Methodology, 11(3), 2002.

\end{thebibliography}

\end{document}
